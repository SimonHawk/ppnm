\documentclass[twocolumn]{article}
\usepackage{graphicx}
\usepackage{amsmath}
\title{\vspace{-3.5cm}Choosing boundry condition for the 0'th order spherical bessel function of the first kind}
\author{S.H.~Albrechtsen}

\begin{document}
\maketitle
All formulas used in this text can be found on wikipedia, see \cite{wikiBesselFunction} and \cite{wikiGammaFunction}.\\

The differential equation for the spherical bessel function can be written as:
\begin{equation}
y'' = -\frac{1}{x^2}(2x y' + (x^2 - n(n+1)))
\end{equation}

And the naive boundry condition one could use to integrate this numerically would be:
\begin{equation}
\textrm{For } x = 0, \quad y = 1,\quad y'=0
\end{equation}
However, it is not possible to numerically evaluate the differential equation at $x=0$. A different starting position is needed, and thereby different boundry condition values must be evaluated.\\

Let's choose the starting point $a$, which is a small number. By choosing a small shift from zero, using a series expansion of the starting conditions becomes a good approximation. However, the solution of the differential equation can obviously not be used to calculate these values.\\

Luckily, the differential equation itself gives a series expansion of the $j_0$ function, which is good for small displacements from 0. This is actually based on the series expansion of the generalized bessel function $J_{\frac{1}{2}}(x)$.\footnote{See the wikipedia page for bessel functions to find more series expansions and do this for other spherical bessen functions than $j_0$} Here the two first terms of the expansion is included:
\begin{align}
j_0(x) & = \sqrt{\frac{\pi}{2x}}J_{1/2}(x)\\
& \approx \sqrt{\frac{\pi}{2x}} \left( \frac{1}{\Gamma(3/2)}(x/2)^{1/2} - \frac{1}{\Gamma(5/2)}(x/2)^{5/2}  \right)
\end{align}
Using $\Gamma(3/2) = \sqrt(\pi)/2$ and $\Gamma(5/2) = 3\sqrt{\pi}/4$:
\begin{align}
j_0(x) & \approx 1 - \sqrt{\frac{\pi}{2x}}\frac{1}{3\sqrt{\pi}/4}\frac{x^{5/2}}{2^{3/2}}\\
& = 1-\frac{1}{3}x^2
\end{align}

Thereby the first derivative of $j_0$ is:
\begin{equation}
j'_0(x) \approx -\frac{2}{3}x
\end{equation}

And thereby, for a given small displacement from 0 called $a$, better boundry condition can be given:
\begin{equation}
\textrm{For } x = a, \quad y = 1 - \frac{a^2}{2},\quad y'= -\frac{2}{3}a
\end{equation}

The accuracy of the boundry condition needs to be at least as good as the accuracy of the differential equation solver. It can be seen that the correction to $y$ is in second order in $a$. Thereby, to have an absolute accuracy $acc$, $a$ needs to satisfy the relation:
\begin{equation}
a \geq \sqrt{acc}
\end{equation}
Of course, choosing a too small $a$ might very well introduce numerical errors, like it would in the differential equation given in the start of this text. If this is the case, the series expansion of $j_0$ must be taken to higher terms, so that the lower limit of $a$ can be a cubic- or higher-order-root of $acc$. Thereby $a$ can be set higher without loss of precision.

\bibliography{references}{}
\bibliographystyle{plain}

\end{document}
